\documentclass[11pt]{scrartcl}
% $Id: Changes-v3.tex 51 2014-06-27 11:05:26Z jan.sieber $
\usepackage[scaled=0.9]{helvet}
\usepackage[T1]{fontenc}
\usepackage[scaled=0.9]{beramono}
\usepackage{amsmath,graphicx,upquote}
\usepackage{gensymb,paralist}
\usepackage{mathpazo}
%\usepackage{eulervm}
%\usepackage[notref,notcite]{showkeys}
%\usepackage[charter]{mathdesign}
\usepackage{color,listings,calc,url}
\typearea{12}
\usepackage[pdftex,colorlinks]{hyperref}
\definecolor{darkblue}{cmyk}{1,0,0,0.8}
\definecolor{darkred}{cmyk}{0,1,0,0.7}
\hypersetup{anchorcolor=black,
  citecolor=darkblue, filecolor=darkblue,
  menucolor=darkblue,pagecolor=darkblue,urlcolor=darkblue,linkcolor=darkblue}
%\renewcommand{\floor}{\operatorname{floor}}
\newcommand{\mt}[1]{\mathrm{#1}}
\newcommand{\id}{\mt{I}}
\newcommand{\matlab}{\texttt{Matlab}}
\renewcommand{\i}{\mt{i}}
\renewcommand{\d}{\mathop{}\!\mathrm{d}}
\renewcommand{\epsilon}{\varepsilon}
\renewcommand{\phi}{\varphi}
\newcommand{\sign}{\operatorname{sign}}
\newcommand{\atant}{\blist{atan2}}
\providecommand{\e}{\mt{e}}
\newcommand{\re}{\mt{Re}}
\newcommand{\im}{\mt{Im}}
\newcommand{\nbc}{n_\mt{bc}}
\newcommand{\gbc}{g_\mt{bc}}
\newcommand{\gic}{g_\mt{ic}}
\newcommand{\nic}{n_\mt{ic}}
\newcommand{\R}{\mathbb{R}}
\usepackage{microtype}

\definecolor{var}{rgb}{0,0.25,0.25}
\definecolor{comment}{rgb}{0,0.5,0}
\definecolor{kw}{rgb}{0,0,0.5}
\definecolor{str}{rgb}{0.5,0,0}
\newcommand{\mlvar}[1]{\lstinline[keywordstyle=\color{var}]!#1!}
\newcommand{\blist}[1]{\mbox{\lstinline!#1!}}
\newlength{\tabw}
\lstset{language=Matlab,%
  basicstyle={\ttfamily\small},%
  commentstyle=\color{comment},%
  stringstyle=\color{str},%
  keywordstyle=\color{kw},%
  identifierstyle=\color{var},%
  upquote=true,%
  deletekeywords={beta,gamma,mesh}%
}
\title{Changes of functionality or code in core DDE-BifTool
  routines from v.~2.03 to v.~3.0}
\author{Jan Sieber}\date{\today}

\begin{document}
\maketitle
% \tableofcontents
% \section{Changes of functionality or code in core DDE-BifTool
%   routines}
% \label{sec:corechanges}
Substantial changes to the following core DDE-BifTool
\cite{ELS01,ELR02,homoclinic,RS07,VLR08} routines were
made. % The updates of the core routines are placed in the folder
% \texttt{ddebiftool\_extra\_psol/}, keeping the original version in the
% folder \texttt{ddebiftool/}.

\paragraph{\blist{psol_jac}}
Enabled vectorization, re-use for Floquet multipliers and vectors, and
nested state-dependent delays \cite{S13}. Additional optional inputs (as
name-value pairs):
\begin{compactitem}
\item \blist{'wrapJ'} (default \blist{true}): if \blist{false} mesh
  gets extended backward and forward in time to cover all delayed time
  points. For \blist{wrapJ==false} no augmentation is done (that is,
  derivatives w.r.t. period or parameters are not calculated) and no
  phase or boundary conditions are appended to output matrix
  \mlvar{J}.
\item \blist{'bc'} (default \blist{true}): controls whether to append
  boundary conditions.
\item \blist{'c_is_tvals'} (default \blist{false}): entries of
  argument \blist{c} are interpreted as the collocation points in the
  full interval (usually \blist{c} is empty or giving the collocation
  points relative to a subinterval). The residual is only calculated
  in these points (incompatible with \blist{'bc'==true}).
\item \blist{'Dtmat'} (default \blist{eye(size(psol_prof,1))}):
  pre-factor in front of time derivative. This permits evaluation of
  algebraic constraints.
\end{compactitem}
Additional outputs:
\begin{compactitem}
\item additional output \mlvar{tT} (array of delays, scaled by period)
\item additional output \mlvar{extmesh} (mesh of time points extended
  back to \blist{-max(tT(:))} if \blist{wrapJ} is \blist{false},
  otherwise equal to argument \blist{mesh}.
\end{compactitem}
The loops are re-arranged to enable a single vectorized call of the
user functions in the new function \mlvar{psol_sysvals}. Moreover, the
evaluation of \blist{xx} at points in $[0,1]$ now uses
\blist{psol_eva} to avoid code repetition.

\paragraph{\blist{p_correc}} Changed to avoid code duplication and
permit finding of zero-crossings of state-dependent delays with
arbitrary levels of nesting. The original contained a large chunk of
code repeating code from \blist{psol_jac} to create a constraint of
the form $\tau_j(t_z)=0$, $\tau'_j(t_z)=0$ for a fixed
$t_z\in[0,1]$. This code has been replaced by an addiitonal function
\blist{delay_zero_cond}, which in turn calls the now more flexible
\blist{psol_jac} to perform the calculations.

\paragraph{\blist{mult_app}}Changed to avoid code duplication.
\begin{compactitem}
\item Has second (optional) output argument \mlvar{eigenfuncs},
  returning the eigenfunctions on the extended mesh \mlvar{extmesh}
  as output by \blist{psol_jac}. The extended mesh (\mlvar{extmesh}
  is the third output in this case.
\item The original version repeated the code for the Jacobian from
  \blist{psol_jac} and appended code for calculating the monodromy
  matrix. These two parts have been replaced by calls to the new more
  flexible \blist{psol_jac} and the manual calculation of the
  monodromy matrix has been replaced by a call to the backslash
  operator.
\item If the delays are negative another (more general, but possibly
  more expensive) algorithm is used. The Jacobian $J$ on
  \mlvar{extmesh}, as output by \mlvar{psol_jac} has dimensions
  $N_r\times
  N_c=\blist{n*(length(mesh)-1)}\times\blist{n*length(extmesh)}$.  We
  solve an augmented and a generalized eigenvalue problem. Let
  $N_\mt{ext}=N_c-N_r$ be the difference between the column and row
  dimensions of the unwrapped Jacobian $J$. Then the generalized
  eigenvalue problem for the eigenpair ($\mu,v)$ is
  \begin{displaymath}
    \begin{bmatrix}
      J\\
      \begin{matrix}
       0_{N_\mt{ext}\times (N_c-N_\mt{ext})}& \id_{N_\mt{ext}} 
      \end{matrix}
    \end{bmatrix}
    v=\mu
    \begin{bmatrix}
      0_{N_r\times N_c}\\
      \begin{matrix}
        \id_{N_\mt{ext}} & 0_{N_\mt{ext}\times (N_c-N_\mt{ext})}
      \end{matrix}
    \end{bmatrix}v\mbox{.}
  \end{displaymath}
  This generalized eigenvalue problem can in principle also be used to
  detect bifurcations of periodic orbits with delays of mixed
  signs. It gives up to numerical round-off errors results that are
  identical to the results from the monodromy matrix used in the
  original code. However, it operates on a pair of large full
  matrices, becoming expensive for large delays or fine
  discretizations.
\end{compactitem}

\paragraph{\blist{p_topsol}} Uses the general routine
\blist{mult_crit} instead of \blist{mult_dbl} and \blist{mult_one} to
compute eigenfunction for critical Floquet multipliers. This avoids
code duplication. Both, \blist{mult_dbl} and \blist{mult_one},
originally duplicated code from the original \blist{psol_jac} and
\blist{mult_app}. The routine \blist{mult_crit} calls \blist{mult_app}
instead.

\paragraph{\blist{psol_eva}, \blist{p_tau}, \blist{p_tsgn},
  \blist{poly_elg}, \blist{poly_del}, \blist{poly_lgr},
  \blist{poly_dla}} Changed to support and speed up call with many
evaluation points.


\paragraph{\blist{df_deriv} and \blist{df_derit}}
Both functions have been amended to enable (pass on) vectorization
such that they can now perform the requested operation for arguments
\blist{xx} of size $n\times (n_\tau+1)\times n_\mt{vec}$ and return
Jacobians of the corresponding shape. They also apply central
difference formulas, making them slower, but potentially more
accurate.  Note that this may result in errors in scripts that
previously worked. For example, if the right-hand side becomes invalid
for certain negative arguments and this argument is close to $0$.

\paragraph{\blist{stst_stabil} and \blist{get_pts_h_new}}
In the computation of eigenvalues of equilibria an a-priori heuristics
estimates where eigenvalues can lie in the complex plane and adjusts
the discretization stepsize accordingly (see \cite{VLR08} for
technical details). The implementation in v.~2.03 resulted in error
messages in various common situations, for example, if the system had
more than $3$ delays, if all delays were zero, or if the estimates
returned only  real parts less than \blist{minimal_real_part}.


\paragraph{Additional auxiliary functions}
\begin{itemize}
\item \blist{p_dot} computes the dot product between two points of
  kind \blist{'psol'}. The function uses the Gauss weights for the
  evaluation of the integral. The options \blist{'free_par_ind'}
  (default empty) and \blist{'period'} (default \blist{false}) set
  which free parameters (non-zero list of indices are included into
  the dot product. The option \blist{'derivatives'} (default
  \blist{[0,0]}) sets hiw often each of the two profiles is
  differentiated before taking the dot product (useful for computing
  products of type $\int_0^1(\dot p(t))^Tq(t)\d t$).
% \item \blist{monodromy_matrix} extracts monomdromy matrix from
%   Jacobian created by \blist{psol_jac}. The code is taken factored out
%   of \blist{mult_app} for possible re-use in other functions.
\item \blist{delay_zero_cond} is a function that has a format suitable
  for use with \blist{method.extra_condition}. It returns the residual
  and the Jacobian for $\tau_j(t_z)=0$ for point types
  \blist{'stst'},\blist{'hopf'} and \blist{'fold'}, and
  $\tau_j(t_z)=0$, $\tau'_j(t_z)=0$ for point type \blist{'psol'}.
\item \blist{VAopX} and \blist{sparse_blkdiag} functions enabling
  vectorized matrix multiplication. For a $n\times m\times p$ array
  \blist{A} and a $m\times q\times p$ array \blist{B}
  \begin{lstlisting}
    C=VAopX(A,B,'*')
  \end{lstlisting}
  gives an $n\times q\times p$ array \blist{C} consisting of the
  matrix products for each of the $p$ stacked matrices.
\item \blist{psol_sysvals} checks if \blist{funcs.x_vectorized} is set
  and performs user function calls (either vectorized or not). This part
  has been factored out of \blist{psol_jac} to make \blist{psol_jac}
  less complex.
\item \blist{dde_set_options} is an auxiliary routine used for
  treatment of optional arguments.
\end{itemize}
{\small\bibliographystyle{unsrt} \bibliography{manual}
}
\end{document}
