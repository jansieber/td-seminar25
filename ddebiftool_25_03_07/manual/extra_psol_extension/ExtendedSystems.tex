\documentclass[11pt]{scrartcl}
\pdfoutput=1
\usepackage[scaled=0.9]{helvet}
\usepackage[scaled=0.9]{beramono}
\usepackage{amsmath,graphicx,upquote}
\usepackage{gensymb,paralist}
%\usepackage{mathpazo}
%\usepackage{eulervm}
%\usepackage[notref,notcite]{showkeys}
\usepackage[charter]{mathdesign}
\usepackage{color,listings,calc,url}
\typearea{11}
\usepackage[pdftex,colorlinks]{hyperref}
\definecolor{darkblue}{cmyk}{1,0,0,0.8}
\definecolor{darkred}{cmyk}{0,1,0,0.7}
\hypersetup{anchorcolor=black,
  citecolor=darkblue, filecolor=darkblue,
  menucolor=darkblue,pagecolor=darkblue,urlcolor=darkblue,linkcolor=darkblue}
%\renewcommand{\floor}{\operatorname{floor}}
\newcommand{\mt}[1]{\mathrm{#1}}
\newcommand{\id}{\mt{I}}
\newcommand{\matlab}{\texttt{Matlab}}
\renewcommand{\i}{\mt{i}}
\renewcommand{\d}{\mathop{}\!\mathrm{d}}
\renewcommand{\epsilon}{\varepsilon}
\newcommand{\sign}{\operatorname{sign}}
\newcommand{\atant}{\blist{atan2}}
\providecommand{\e}{\mt{e}}
\newcommand{\re}{\mt{Re}}
\newcommand{\im}{\mt{Im}}
\newcommand{\nbc}{n_\mt{bc}}
\newcommand{\gbc}{g_\mt{bc}}
\newcommand{\gic}{g_\mt{ic}}
\newcommand{\nic}{n_\mt{ic}}
\newcommand{\nbp}{n_\mt{bp}}
\newcommand{\nint}{n_\mt{int}}
\newcommand{\neqs}{n_\mt{eqs}}
\newcommand{\nd}{n_\mt{cd}}
\newcommand{\R}{\mathbb{R}}
\DeclareMathOperator{\diag}{diag}
\usepackage{microtype}

\definecolor{var}{rgb}{0,0.25,0.25}
\definecolor{comment}{rgb}{0,0.5,0}
\definecolor{kw}{rgb}{0,0,0.5}
\definecolor{str}{rgb}{0.5,0,0}
\newcommand{\mlvar}[1]{\lstinline[keywordstyle=\color{var}]!#1!}
\newcommand{\blist}[1]{\mbox{\lstinline!#1!}}
\newlength{\tabw}
\lstset{language=Matlab,%
  basicstyle={\ttfamily},%
  commentstyle=\color{comment},%
  stringstyle=\color{str},%
  keywordstyle=\color{kw},%
  identifierstyle=\color{var},%
  upquote=true,%
  deletekeywords={beta,gamma}%
}

\title{Extended systems for 
  Delay-differential equations as 
  implemented in the extensions to  DDE BifTool}
\author{Jan Sieber}\date{}
\begin{document}
\maketitle
\tableofcontents
\begin{abstract}
  \noindent
  \textbf{\textsf{Abstract}}\quad This document lists the extended
  systems defining bifurcations of periodic orbits in delay
  differential equations (DDEs) with finitely many point delays. The
  delays are permitted to be system parameters or dependent on the
  state at arbitrary points in the history. All systems are
  boundary-value problems with periodic boundary conditions.

  The proposed systems have been implemented as extensions to
  DDE-BifTool, a numerical bifurcation analysis package for DDEs for
  Matlab or Octave. The document contains also practical details of
  the implementation, pointing to relevant routines.
\end{abstract}


\section{Background}
\label{sec:background}
The references \cite{MDO09,G00,Doed07,DCFKSW98} give background
information on the defining systems for bifurcations of periodic
orbits in ODEs.  The systems defining the periodic-orbit bifurcations
for constant or state-dependent delay can be written as periodic
boundary-value problems (BVPs) of DDEs that can be tracked using the
\blist{'psol'} routines of
DDE-BifTool\footnote{\url{http://twr.cs.kuleuven.be/research/software/delay/ddebiftool.shtml}}. The
theory for bifurcations of periodic orbits in DDEs with constant delay
is covered in textbooks \cite{S89,Hale93,Diek95}. Details of the
algorithms and theory behind DDE-BifTool are discussed in
\cite{ELS01,ELR02,SER02,RS07,VLR08}. The review \cite{RS07} also
refers to
\texttt{knut}\footnote{\url{http://gitorious.org/knut/pages/Home}}, an
alternative C++ package with a stand-alone user interface. The review
\cite{HKWW06} lists difficulties and open problems occuring with DDEs
with state-dependent delays. However, periodic BVPs can be reduced to
smooth finite-dimensional systems of algebraic equations
\cite{S12}. This means that the general bifurcation theory for DDEs
with state-dependent delay works as expected as far as the Hopf
bifurcation, regular branches of periodic orbits, the period doubling
bifurcation and Arnol'd tongues branching off at resonant points at a
torus bifurcation are concerned.

In the following all DDEs are assumed to have periodic boundary
conditions on the interval $[0,1]$.

\section{Folds of periodic orbits --- constant delays}
\label{sec:ext:fold}
In DDE-BifTool the system defining a periodic orbit is the periodic
DDE with $n_\tau$ delays
\begin{align}
  \label{eq:po:dde}
  0&=\frac{M}{T}\dot x(t)-f\left(x(t),\cfrac{x^{(k_1)}(t-\tau_1/T)}{T^{k_1}},\ldots,\cfrac{x^{(k_1)}(t-\tau_{n_\tau}/T)}{T^{k_{n_\tau}}},p\right)\mbox{,}\\
  \label{eq:po:phas}
  0&=\int_0^1\dot x_\mathrm{ref}(t)^Tx(t)\d t
\end{align}
where the unknowns are $x\in C_\mathrm{p}^n:=C_\mathrm{per}([0,1];R^n)$ (space
of periodic functions on $[0,1]$) and $T$. Using the operators
\begin{align*}
  [E^\ell_{\tau,T}x](t)&:=\cfrac{x^{(\ell)}(t-\tau/T)}{T^\ell},&
  F(x_0,\ldots x_{n_\tau},p)(t)&:=f(x_0(t),\ldots x_{n_\tau}(t),p),
\end{align*}
equation \eqref{eq:po:dde} reads as an identity for functions in $C^n_\mathrm{p}$ (defining $\tau_0=0$, $\ell_0=0$):
\begin{align}
  \label{eq:po:dde:op}
  0&=ME^1_{0,T}x-F\left(\left(E^{\ell_k}_{\tau_k,T}x\right)_{k=1}^{n_\tau},p\right)\mbox{.}
\end{align}
The linearizations of $E$ w.r.t. $\tau$ and $T$ are
\begin{align*}
  \partial_\tau E^\ell_{\tau,T}&:=-E^{\ell+1}_{\tau,T},&
  \partial_T E^\ell_{\tau,T}&=\frac{1}{T}\left[\tau E^{\ell+1}_{\tau,T}-\ell E^\ell_{\tau,T}\right]\mbox{.}
\end{align*}
The criterion for a fold is that system
\eqref{eq:po:dde:op},\,\eqref{eq:po:phas} has a simple singularity in
$x\in C^n_\mathrm{p}$, that is, its linearization has a simple
nullvector. Some problems (for example, with rotational symmetry) have
additional free parameters that we consider part of $p$. Let us assume
that those free $n_q$ parameters are selected by $n_p\times n_q$
matrix $\Delta_p$, such that the derivative of $f$ with respect to the
free parameters is $\partial_pf(\cdot)\Delta_p$. Hence ,the nullvector
is $(\delta_x,\delta_T,\delta_p)\in C_p^n\times\R\times\R^{n_q}$. This nullvector
satisfies the linear system of equations:
\begin{align}
  \label{eq:fold:dde+dt}
  0&=ME^1_{0,T}\delta_x-\sum_{k=0}^{n_\tau}\partial_kFE^{\ell_k}_{\tau_k,T}\delta_x+
  \frac{\delta_T}{T}\cdot\left[
    \sum_{k=0}^{n_\tau}\partial_{k}F \left[\ell_kE^{\ell_k}_{\tau_k,T}-\tau_kE^{\ell_k+1}_{\tau_k,T}\right]-ME^1_{0,T}\right]x-\partial_pF\Delta_p\delta_p\mbox{,}\\
  \label{eq:fold:phas0}
  0&=\int_0^1\dot x_\mathrm{ref}(t)^T\delta_x(t)\d t
\end{align}
Note that the factor for $\delta_T$ (in brackets) in
\eqref{eq:fold:dde+dt} is the derivative of the original DDE
\eqref{eq:po:dde} with respect to $T$. In \eqref{eq:fold:dde+dt} the terms $\partial_kF_*$ stand for (using $\partial_kf$ for the derivative w.r.t. the argument $[E^{\ell_k}_{\tau_k,T}x](t)$ and $\partial_pf$ for the derivative of $f$ w.r.t. parameter $p$)
\begin{align}\label{eq:ftdef}
  [\partial_kFv](t)&=\partial_kf\left(\cdot\right)v(t)\mbox{,}&
  [\partial_pFq](t)&=\partial_pf\left(\cdot\right)q\mbox{,\ where}&
  (\cdot)&=\left(\left(\frac{x^{(\ell_k)}\left(t-\frac{\tau_k}{T}\right)}{T^{\ell_k}}\right)_{k=1}^{n_\tau}\hspace*{-1ex},p\right)
\end{align}
\paragraph{Extended system}
\begin{table}[ht]
  \centering
  \begin{tabular}[t]{l@{\qquad}l}\hline\noalign{\medskip}
    \textbf{Position} in \blist{parameter} & \textbf{parameter}
    \\\noalign{\medskip}
    \blist{1:npar} & user system parameters\\
    \blist{npar+1} & $\beta:=\delta_T/T$\\
    %\blist{npar+2} & parameter $T$ enforced to be equal to period\\
    \blist{npar+1+nnull} & additional derivatives $\delta_p$ w.r.t. other free parameters\\
    \blist{npar+1+nnull+(1:ntau)} & $\tau_{n_\tau+k}=\tau_k$ ($k=1,\ldots,n_\tau$), delays for $x^{(\ell_k+1)}$\\\noalign{\medskip}\hline
  \end{tabular}
  \caption{Parameters of extended system for fold 
    (\blist{ntau=length(sys_tau())}$=n_\tau$)}
  \label{tab:foldpars}
\end{table}
The function variables (stored in \blist{point.profile}) are $(x,\delta_x)\in
C_p^{2n}=C_p^n\times C_p^n$. Inside \blist{pfuncs.sys_rhs} they are
accessed as \blist{x=xx(1:n,:)} and \blist{v=xx(n+1:end,:)}. The
system parameters of the extended system, assuming that the
user-defined system has \blist{npar} parameters, are given in
Table~\ref{tab:foldpars}.  The DDE defined by \blist{sys_rhs_POEV1(xx,par)}
consists of (using the convention $\tau_{0}=0$, $\ell_0=0$, $\beta=\delta_T/T$)
\begin{align}
    \label{eq:po:impdde}
  M\dot x(t)&= f\left((x^{(\ell_k)}(t-\tau_k))_{k=0}^{n_\tau},p\right)\mbox{,}\\
    \label{eq:fold:impdde}
    M\dot v(t)&=\partial f\left((x^{(\ell_k)}(t-\tau_k))_{k=0}^{n_\tau},p\right)\left[(z_k(t))_{k=0}^{n_\tau},\Delta_p\delta_p\right]+\beta M\dot x(t)\mbox{,\quad where}\\
    \nonumber
    z_k(t)&=v^{(\ell_k)}(t-\tau_k)+\beta\tau_kx^{(\ell_k+1)}(t-\tau_k)-\beta\ell_kx^{(\ell_k)}(t-\tau_k).
  \end{align}
  Note that there are only $2n_\tau$ delays, since ther term
  $\tau_0=0$ in \eqref{eq:fold:impdde}. The term
  $\partial f(\ldots)[(z_k)_{k=1}^{n_\tau},q]$ is the directional
  derivative of $f$ in
  $(x(t),x^{(\ell_1)}(t-\tau_1),\ldots,x^{(\ell_{n_\tau})}(t-\tau_{n_\tau}),p)$. It
  has $n_\tau+1$ linear arguments $z_k$ of length $n_x$ and $1$ linear
  argument $q$ of length $n_p$. The parameters in its argument \blist{par}
  are ordered as in Table~\ref{tab:foldpars}.  The additional
  $2+n_{q\cap \tau}$ system conditions implemented in
  \blist{sys_cond_POfold} are:
\begin{align}
  \label{eq:foldvtv1}
  0&=\int_0^1 v(t)^Tv(t)\d t+T^2\beta^2+\sum_{i=1}^{n_q}q_i^2-1\\
  0&=\int_0^1 \dot x_\mathrm{ref}(t)^Tv(t)\d t \label{eq:fold:phas1}\\
  0&=\tau_{k+n_\tau}-\tau_k \mbox{,\quad for delays $k$ that are free parameters.}\label{eq:fold:taurel}
\end{align}
An auxiliary routine \blist{p_dot} is included in the folder
\texttt{ddebiftool} to compute scalar products between \blist{'psol'}
structures of the type \eqref{eq:fold:phas1} (not directly called by
the user but may be useful in other problems).

\paragraph{Initialization}
The Jacobian $J$ of the linearization of
\eqref{eq:po:dde}--\eqref{eq:po:phas} in a point \blist{x} can be
obtained with the DDE-BifTool routine \blist{dde_psol_jac} (calling it
with free parameters in the range of $\Delta_p$:
\blist{free_par=nullparind(:,1)}). The phase condition of the
periodicBVP is included in \blist{dde_psol_jac}. Linearizations of
additional $q$ \blist{sys_cond} type conditions must be appended. If
$J=USV^T$ is the singular-value decomposition of $J$ then
\blist{V(:,end)} is the approximate nullvector of $J$. This column
vector only has to be reshaped and rescaled to achieve
\eqref{eq:foldvtv1}:
\begin{lstlisting}
  v.profile=reshape(V(1:size(x.profile),end),size(x.profile));
  v.parameter=0*x.parameter;
  extrapars=[ip.beta,ip.nullparind(:,1)'];
  v.parameter(extrapars)=V(size(x.profile)+1:end,end);
  v.parameter(ip.beta)=v.parameter(ip.beta)/x.period;
  vnorm=sqrt(p_dot(v,v,'free_par_ind',extrapars);
  v.parameter=v.parameter/vnorm;
  v.profile=v.profile/vnorm;
  x.profile(dim+(1:dim),:)=v.profile;
  x.parameter(ip.beta)=v.parameter(ip.beta);
  x.parameter(ip.nullparind(:,2))=v.parameter(ip.nullparind(:,1));
\end{lstlisting}
Then \blist{x.profile(dim+(1:dim),:)} is the approximation for $v(t)$,
\blist{x.parameter(ip.beta)} is the approximation for $\beta$, and
\blist{x.parameter(ip.nullparind(:,2))} is the approximation for $q$.
\section{Period doublings and torus bifurcations --- constant
  delays}
\label{sec:ext:tr}
The extended system has the function components $x$, $u$ and $v$ (all
in $C_p^n$) and the additional parameters $\omega$ (angle of critical
Floquet multipiers) and $T_\mathrm{copy}$ (equal to the period of the
orbit). The linearized system is written in Floquet exponent
form. This means that, if $\exp(\i \omega\pi)$ is the Floquet multiplier
and $z$ the corresponding eigenfunction (satisfying
$z(1)=z(0)\exp(\i\omega\pi)$), then
\begin{displaymath}
  u(t)+\i v(t)=\exp(-\i\omega\pi t/T)\,z(t)
\end{displaymath}
such that $u$ and $v$ are periodic (and real). The linearized system
in Floquet exponent form (as implemented in \blist{sys_rhs_TorusBif}) is
\begin{align}
  \label{eq:tr:u}
  \dot u(t) &=\phantom{-}\frac{\pi\omega}{T}v(t)+
  \sum_{j=0}^{n_\tau}\partial_{j+1}f(t)\left[\phantom{-}\cos(\pi\omega\tau_j/T)u(t-\tau_j)+
  \sin(\pi\omega\tau_j/T)v(t-\tau_j)\right]\\
  \label{eq:tr:v}
  \dot v(t) &=-\frac{\pi\omega}{T}u(t)+
  \sum_{j=0}^{n_\tau}\partial_{j+1}f(t)\left[-\sin(\pi\omega\tau_j/T)u(t-\tau_j)+
  \cos(\pi\omega\tau_j/T)v(t-\tau_j)\right]\mbox{.}
\end{align}
Again, we used the convention $\tau_0=0$ and $\partial_kf(t)$ as
defined in \eqref{eq:ftdef}.  The additional system conditions (in
\blist{sys_cond_TorusBif}) are
\begin{align}
  0&=\int_0^1 u(t)^Tu(t)+v(t)^Tv(t)\d t-1\label{eq:vtvutu:norm}\\
  0&=\int_0^1 u(t)^Tv(t)\d t\label{eq:vtu:orth}\\
  0&=\blist{point.parameter(npar+2)-point.period} \label{eq:tr:tcopy}
\end{align}
where \blist{npar} is the number of parameters defined by the user.

\paragraph{Initialization}
The auxiliary routine \blist{mult_crit} extracts the critical Floquet
multiplier $\mu$ and the corresponding eigenfunction $z(t)$ for the
approximate bifurcation point \blist{x}. Then the periodic Floquet
mode $y(t)$ is obtained as
\begin{displaymath}
y(t)=y_r(t)+\i y_i(t)=\exp(-\log(\mu)t)\,z(t)\mbox{,}
\end{displaymath}
where $t\in[0,1]$ is already rescaled, and
$\omega=\blist{atan2(imag(mu),real(mu))}/\pi$. Finally, one has to
rescale $y$ to make it of unit length, and make its real and imaginary
part orthogonal. Define
% r=1/sqrt(utu+vtv);
% gamma=atan2(2*utv,vtv-utu)/2;
% qr=r*(upoint.profile*cos(gamma)-vpoint.profile*sin(gamma));
% qi=r*(upoint.profile*sin(gamma)+vpoint.profile*cos(gamma));
\begin{align*}
  r&=\sqrt{\langle y_r,y_r\rangle+\langle
    y_i,y_i\rangle}\mbox{,\quad and }\\
  \gamma&=\frac{1}{2}\blist{atan2}\,(2\langle
  y_r,y_i\rangle,\langle
  y_r,y_r\rangle-\langle y_i,y_i\rangle)\mbox{,\quad then}\\
  u(t)&=[y_r(t)\cos\gamma-y_i\sin\gamma]/r\mbox{,}\\
  v(t)&=[y_r(t)\sin\gamma+y_i\cos\gamma]/r
\end{align*}
satisfy $\langle u,v\rangle=0$ and $\langle u,u\rangle+\langle
v,v\rangle=1$, giving the profiles of $u$ and $v$, needed for the
extended system.

\section{Systems with state-dependent delays}
\label{sec:sd-dde}
For state-dependent delays the user may provide the argument \blist{'sys_tau_seq'}, for example,
\begin{lstlisting}
  funcs=set_funcs(...,'sys_tau_seq',{1:3,4},'sys_ntau',@()4);
\end{lstlisting}
to indicate that \blist{sys_tau} can be called as
\blist{sys_tau(1:3,xx0,par)} with \blist{xx0}=$x(t)$ of format
$n_x\times1$, returning the first $3$ delays, and
\blist{sys_tau(4,xx2,par)} with \blist{xx1} of format $n_x\times4$
(consisting of
$(x(t),x^{\ell)}(t-\tau_1),x^{(\ell)}(t-\tau_2),x^{(\ell)}(t-\tau_3)$)
for some order $\ell\geq0$.  This possibility of returning multiple
delays at once makes the treatment of a large number of
state-dependent delays (approximating distributed delays)
computationally feasible. If \blist{'sys_tau_seq'} is not provided, it
defaults to \blist{num2cell(1:sys_ntau())}.

For this reason we unify the notation by accepting the entire introduce a mildly different notation for the
system defining a periodic orbit by grouping the elements of
\blist{'sys_tau_seq'} into vectors of length $m_k$, each corresponding
to $m_k$ delays with the same arguments, with $m_0=1$ and
$\sum_{k=1}^{m_\tau}m_k=n_\tau$:
\begin{align}
  \label{eq:po:sddde}
  0&=M\frac{\dot x(t)}{T}-f\left([x_0,\ldots,x_{m_\tau}],p\right)\mbox{,\quad where $x_k\in\R^{n_x\times m_k}$,}\\
    \label{eq:po:sdxk}
  x_k&=\left(x^{(\ell_k)}\left(t-\cfrac{\tau_{k,j}([x_0,\ldots,x_{k-1}],p)}{T}\right)\big/T^{\ell_k}\right)_{j=1}^{m_k}\mbox{,}\\
  \nonumber
  \tau_0&=0,\ m_0=1,\  k=0,\ldots,m_\tau,\quad \sum_{j=1}^{m_\tau}m_k=n_\tau,
\end{align}
augmented with the phase condition \eqref{eq:po:phas} to define the
period $T$. In \eqref{eq:po:sdxk} the delays are functions
\begin{align*}
\tau_k:&\R^{n_x\times \mu_k}\times\R^{n_p}\mapsto\R^{m_k}\mbox{,\ where\quad}
\mu_k=\sum_{j=0}^{k-1}m_k\mbox{,\ and\ }0=\mu_0<\mu_1<\ldots<\mu_{m_\tau}=n_\tau+1.
\end{align*}
In the extreme case that all delays only depend on $x(t)$ and $p$, we
have the arguments
\begin{lstlisting}
  funcs=set_funcs(...'sys_tau_seq',{1:ntau},'sys_ntau',@()ntau,...);
\end{lstlisting}
where \blist{sys_tau(1:ntau,xx0,par)} is called once with \blist{xx0} of size
$n_x\times1$, while \blist{sys_rhs(xx,par)} is called with \blist{xx}
of size $n_x\times(1+n_\tau)$
($[x(t),x(t-\tau_1(x(t),p)),\ldots,x(t-\tau_{n_\tau}(x(t),p))]$), such
that $m_\tau=1$, $m_0=0$, $m_1=n_\tau$.

\paragraph{Remark} Every system of form
\begin{align}
  \label{eq:po:sddde}
  0&=M\frac{\dot x(t)}{T}-f\left([x_0,\ldots,x_{n_\tau}],p\right)\mbox{,\quad where $x_0=x(t)$,}\\
    \label{eq:po:sdxk}
    x_k&=x^{(\ell_k)}\left(t-\tau_k([x_0,\ldots,x_{k-1}],p)/T\right)\big/T^{\ell_k}\mbox{,\ ($k=1,\ldots,n_\tau$)}
\end{align}
(the default scenario of
\eqref{eq:po:sddde},\,\eqref{eq:po:sdxk}, where \blist{'sys_tau_seq'}
equals \blist{num2cell(1:sys_ntau())} can be transformed into a
system where all delays depend on $x(t)$ and parameter $p$ only. To do this one
introduces the delays explictly as additional time-dependent
variables $x_{n_x+1}(t),\ldots,x_{n_x+n_\tau}(t)$ and adding the
$n_\tau$ algebraic equations
\begin{align}
  0&=\tau_1(x_{1,\ldots,n_x}(t),p)-x_{n_x+1}(t)\mbox{, and for $k=2\ldots,n_\tau$}\\
  0&=\tau_k([x_{1,\ldots,n_x}(t),x_{1,\ldots,n_x}(t-x_{n_x+1}(t)),\ldots,
  x_{1,\ldots,n_x}(t-x_{n_x+k-1}(t))],p)-x_{n_x+k}(t).
\end{align}
This reduces the complexity of the format of calling \blist{sys_tau},
but adds $n_\tau$ variables to the system, which may not be
appropriate if the number of delays is large.

\paragraph{Operator notation}
We correspondingly generalize our
notation $E$ to multi-column vectors $\hat{E}$. For a vector
$\tau\in\R^m$, integer $\ell\geq0$, scaling $T>0$,
\begin{align*}
  [\hat{E}^\ell_{\tau,T}x](t)&:=\left(x^{(\ell)}(t-\tau_1/T),\ldots,x^{(\ell)}(t-\tau_m/T)\right)/T^\ell\in\R^{n_x\times m},\\
  F(x_0,\ldots x_{m_\tau},p)(t)&:=f(x_0(t),\ldots x_{m_\tau}(t),p)\in\R^{n_x},
\end{align*}
the linearizations of $\hat{E}$ are identical to those in the constant delay case
\begin{align*}
  \partial_\tau \hat{E}^\ell_{\tau,T}x(t)\delta_\tau&:=-\hat{E}^{\ell+1}_{\tau,T}x(t)\diag(\delta_\tau)\in\R^{n_x\times m}\mbox{\quad for $\delta_\tau\in\R^{m\times1}$, $\diag(\delta_\tau)\in\R^{m\times m}$,}\\
  \partial_T \hat{E}^\ell_{\tau,T}x(t)&=\frac{1}{T}\left[\hat{E}^{\ell+1}_{\tau,T}x(t)\diag(\tau)-\ell \hat{E}^\ell_{\tau,T}x(t)\right]\in\R^{n_x\times m}\mbox{.}
\end{align*}
The operation $\diag(\tau)$ for a vector $\tau$ results in a diagonal
matrix $D$ with diagonal entries $D_{i,i}=\tau_i$. For a function
$\tau:\R^{n_x\times\mu}\times\R^{n_p}\to\R^m$, we have the
linearization of
\begin{align*}
  \hat{E}^\ell_{\tau(y,p),T}x(t)=\left(x^{(\ell)}(t-\tau_1/T)/T^\ell,\ldots,x^{(\ell)}(t-\tau_m/T)/T^\ell\right)\in\R^{n_x\times m}
\end{align*}
with respect to deviation $(\delta_x,\delta_y,\delta_p,\delta_T)$:
\begin{multline*}
  \partial\hat{E}^\ell_{\tau(y,p),T}x[\delta_x,\delta_y,\delta_p,\delta_T](t)=
  \hat{E}^\ell_{\tau(y,p),T}\delta_x(t)-\ell\frac{\delta_T}{T}
  \hat{E}^\ell_{\tau(y,p),T}x(t)+
  \\
  \hat{E}^{\ell+1}_{\tau(y,p),T}x(t)\left[\diag\left(\tau(y,p)\frac{\delta_T}{T}-\partial\tau(y,p)[\delta_y,\delta_p]\right)\right].
\end{multline*}
Using the oprators $\hat{E}$ and $F$ the differential equation reads
\begin{align}
  \label{eq:po:sddde:EF}
  0&=M\hat{E}^1_{0,T}x-F\left([x_0,\ldots,x_{m_\tau}],p\right)\mbox{,\quad where\quad}
  x_k=\hat{E}^{\ell_k}_{\tau_k([x_0,\ldots,x_{m_{k-1}}],p),T}x\mbox{,}
\end{align}
with $\ell_0=0$, $\tau_0=0$ for $k=0\ldots,m_\tau$.
\subsection{The general variational problem}
\label{sec:vp}
The variational problem of \eqref{eq:po:sddde}---\eqref{eq:po:sdxk}
linearized in $(x(\cdot),T,p)$ with respect to
$(\delta_x(\cdot),\delta_T,\delta_p)$ is:
\begin{align}\allowdisplaybreaks
  0=&M\hat{E}^1_{0,T}\delta_x-M\hat{E}^1_{0,T}x\frac{\delta_T}{T}-\partial F(y_{m_\tau},p)[Y_{m_\tau},\delta_p]\label{eq:vp}
  \mbox{,\qquad where for $k=0,\ldots,m_\tau$,}\\
    x_k=&\hat{E}^{\ell_k}_{\tau_k(y_{k-1},p),T}x\mbox{,}\quad y_k=[x_0,\ldots,x_k],\quad Y_k=[X_0,\ldots,X_k]
  \mbox{,}\nonumber\\
  X_k=&\hat{E}^{\ell_k}_{\tau_k(y_{k-1},p),T}\delta_x-\ell_k\frac{\delta_T}{T}
  \hat{E}^{\ell_k}_{\tau_k(y_{k-1},p),T}x+\nonumber\\
  &\quad\hat{E}^{\ell_k+1}_{\tau_k(y_{k-1},p),T}x\left[\diag\left(\tau_k(y_{k-1},p)\frac{\delta_T}{T}-\partial\tau_k(y_{k-1},p)[Y_{k-1},\delta_p]\right)\right].\label{eq:vp:qk}
\end{align}
Since $\ell_0=0$, $\tau_0=0$ and $\partial\tau_0=0$, we have in
particular that $x_0=E^0_{0,T}x=x$ and
$X_0=\hat{E}^0_{0,T}\delta_x=\delta_x$. At each time $t\in[0,1]$ the
$X_k(t)$ have shape $n_x\times m_k$ and $Y_k(t)$ have shape
$n_x\times\mu_k=n_x\times(m_0+\ldots+m_k)$.  The matrix functions
$t\mapsto X_k(t)\in\R^{n_x\times m_k}$ are the directional derivatives
of $t\mapsto x_k(t)\in\R^{n_x\times m_k}$ in direction\
$(t\mapsto \delta_x(t),\delta_T,\delta_p)$.

We observe that the definition of $X_k$ recursively requires
$Y_{k-1}=[x_0,\ldots,x_{k-1}]$ such that the $X_k$ could be computed
iteratively over $k$ from $0$ to $m_\tau$. Alternatively, introducing the notation
\begin{align*}
  \hat{\tau}_k(y_{m_\tau},p)&=\hat{\tau}_k(y_k,p),&
  \partial\hat{\tau}_k(y_{m_\tau},p)[Y_{m_\tau},\delta_p]&=\partial\tau_k(y_k,p)[Y_k,\delta_p],\\
  \hat{\tau}&=[\hat{\tau}_0,\ldots,\hat{\tau}_{m_\tau}],&
  \hat{E}^\ell_{\hat{\tau}(y,p),T}x&=\left[\hat{E}^{\ell_l}_{\hat{\tau}_k(y,p),T}x\right]_{k=0}^{m_\tau},
\end{align*}
(permitting the longer argument vectors $y_{m_\tau}$ and $Y_{m_\tau}$
to be passed on to $\hat{\tau}_k$ and $\partial\hat{\tau}_k$ such that
all $\hat{\tau}_k$ have the same arguments), we observe that
$Y_{m_\tau}$ satisfies the linear equation
\begin{align}
  \label{eq:Yk:lin}
  [I-\Theta]Y&=Y_\mathrm{rhs}\mbox{\quad where}\\
  \nonumber%\label{eq:Yk:lin:rhs}
  Y_\mathrm{rhs}&=\hat{E}^\ell_{\hat{\tau}(y,p)}\delta_x+\left[\hat{E}^{\ell+1}_{\hat{\tau}(y,p)}x\diag(\hat{\tau}(y,p))-\hat{E}^\ell_{\hat{\tau}(y,p)}x\diag(\ell)\right]\frac{\delta_T}{T}-\hat{E}^{\ell+1}_{\hat{\tau}(y,p)}x\partial\hat{\tau}(y,p)[0,\delta_p]\mbox{,}\\
  \nonumber%\label{eq:Yk:lin:mat}
  \Theta Y&=-\hat{E}^{\ell+1}_{\hat{\tau}(y,p)}x\partial\hat{\tau}(y,p)[Y,0]\mbox{,\ with the abbreviations\quad}
  Y=Y_{m_\tau},\quad y=y_{m_\tau}.
\end{align}
When the solution $x$ is discretized into $\nint$ collocation
intervals with collocation polynomial order $\nd$, such that we have
$\nbp=\nint\nd+1$ base time points (typically in $[0,1]$, but for
stability computations the base interval is extended to
$[-\tau_{\max},1]$)) and $\neqs=\nint\nd$ collocation time points (where equations are imposed), then the dimension of the terms involved is
\begin{align*}
  \hat{E}^\ell_{\hat{\tau}(y,p)}&:n_x\neqs (n_\tau+1)\times n_x\nbp,\\
  \hat{E}^{\ell+1}_{\hat{\tau}(y,p)}x\diag(\hat{\tau}(y,p))-\hat{E}^\ell_{\hat{\tau}(y,p)}x\diag(\ell)&:n_x\neqs(n_\tau+1)\times1\mbox{\quad ($\diag(\cdot)$ applied to $3$rd dim.)}\\
  \hat{E}^{\ell+1}_{\hat{\tau}(y,p)}x\partial\tau(y,p)[0,(\cdot)]&:n_x\neqs(n_\tau+1)\times n_\mathrm{fp},\\
  \Theta&:n_x\neqs(n_\tau+1)\times n_x\neqs(n_\tau+1).
\end{align*}
In the dimension specification the order of terms in the products
indicates the ordering of the vectors/matrices. The matrix $\Theta$ is
nilpotent with $\Theta^{m_\tau+1}=0$ such that the iteration given in
\eqref{eq:vp:qk} is equivalent to a backsubstitution, or, if one uses
matrix free directional derivatives for $\partial\hat{\tau}$,
\begin{align*}
  Y=\sum_{k=0}^{m_\tau}\Theta^kY_\mathrm{rhs}.
\end{align*}
Using the above notation, the general variational problem can be written as
\begin{align}\allowdisplaybreaks
  0=&M\hat{E}^1_{0,T}\delta_x-M\hat{E}^1_{0,T}x\frac{\delta_T}{T}-\partial F(y,p)[Y,\delta_p]\label{eq:genvp}
  \mbox{,\quad where\quad}Y=[I-\Theta]^{-1}Y_\mathrm{rhs}=\sum_{k=0}^{m_\tau}\Theta^kY_\mathrm{rhs},\\
  \label{eq:genvp:lin:mat}  
  \Theta&=-\hat{E}^{\ell+1}_{\hat{\tau}(y,p)}x\partial_y\hat{\tau}(y,p)\\
  \label{eq:genvp:lin:rhs}  
  Y_\mathrm{rhs}&=\hat{E}^\ell_{\hat{\tau}(y,p)}\delta_x+\left[\hat{E}^{\ell+1}_{\hat{\tau}(y,p)}x\diag(\hat{\tau}(y,p))-\hat{E}^\ell_{\hat{\tau}(y,p)}x\diag(\ell)\right]\frac{\delta_T}{T}-\hat{E}^{\ell+1}_{\hat{\tau}(y,p)}x\partial_p\hat{\tau}(y,p)\delta_p\mbox{.}% \\
  % \label{eq:genvp:lin:mat}
  % \Theta &=-\hat{E}^{\ell+1}_{\hat{\tau}(y,p)}x\partial_y\tau(y,p)
  % x_k=&\hat{E}^{\ell_k}_{\tau_k(y_{k-1},p),T}x\mbox{,}\quad y_k=[x_0,\ldots,x_k],\quad Y_k=[X_0,\ldots,X_k]
  % \mbox{,}\nonumber\\
  % X_k=&\hat{E}^{\ell_k}_{\tau_k(y_{k-1},p),T}\delta_x-\ell_k\frac{\delta_T}{T}
  % \hat{E}^{\ell_k}_{\tau_k(y_{k-1},p),T}x+\nonumber\\
  % &\quad\hat{E}^{\ell_k+1}_{\tau_k(y_{k-1},p),T}x\left[\diag\left(\tau_k(y_{k-1},p)\frac{\delta_T}{T}-\partial\tau_k(y_{k-1},p)[Y_{k-1},\delta_p]\right)\right].\label{eq:vp:qk}
\end{align}
The constant-delay case is the special case where $m_\tau=0$ such that $Y=Y_\mathrm{rhs}$.
\paragraph{Practical considerations} Continuation of periodic orbit
bifurcations computes the delays $\tau_{j,k}$ and calls the right-hand
side $f$ at $(x(t-\tau_{k,0}),\ldots,x(t-\tau_{k,n_\tau}))$. The
function \mlvar{tauSD_ext_ind} creates an array \mlvar{xtau_ind} which
sorts the delays as follows: $\tau_{0,k}=\tau_{k,0}=\blist{tau(k)}$
for $1\leq k\leq n_\tau$,
$\tau_{j,k}=\blist{tau((j-1)*ntau+ntau+1+k)}$ for $j\geq1$ and
$k=1,\ldots,n_\tau$. Inside the array \mlvar{xx}, which is the first
argument of \blist{sys_rhs} and \blist{sys_tau},
$(x(t-\tau_{k,0}),\ldots,x(t-\tau_{k,n_\tau}))$ can be found as
\blist{xx(:,xtau_ind(k+1,:))}



\section{Fold of periodic orbits --- state-dependent delays}
\label{sec:sd:pofold}
The function \mlvar{sys_rhs_SD_POfold} extends the user-defined
nonlinear problem with the variational problem
\eqref{eq:vp}---\eqref{eq:vp:qk} restricted to $q=0$. As a
user-defined function it assumes that its argument $x(\cdot)$ has
period $T$ (thus, all delays and derivatives are re-scaled by
$T$). Repeating all definitions, \mlvar{sys_rhs_SD_POfold}
implements the following system:
\begin{equation}\label{eq:sd:pofold}
  \begin{aligned}
    \dot x(t)&=f\left(x_0,\ldots,x_{n_\tau},p\right)\mbox{,}\\
    \dot v(t)=&\frac{\beta}{T}f(t)+
    \sum_{k=0}^{n_\tau}\partial_{x,k}f(t)[V_k+B_k]
    \mbox{,\qquad where}\\
    V_0=&v(t)\mbox{,}\quad B_0=0\mbox{\quad ($\in\R^n)$,}\\
    V_k=&v(t-\tau_k(t))-f_k(t)\sum_{j=0}^{k-1}\partial_{x,j}
    \tau_k(t)V_j\mbox{,\quad($k=1\ldots n_\tau$)}\\
    B_k=&f_k(t)\left[\frac{\tau_k(t)}{T}\beta-
      \sum_{j=0}^{k-1}\partial_{x,j}
      \tau_k(t)B_j\right]\mbox{,\quad($k=1\ldots n_\tau$).}
  \end{aligned}
\end{equation}
System~\eqref{eq:sd:pofold} uses the notations
\begin{equation}\label{eq:sd:abbrevs}
  \begin{aligned}
    x_k&=x(t-\tau_k(x_0,\ldots,x_{k-1},p))&&\mbox{\quad ($\tau_0=0$),
       $k=0,\ldots,n_\tau$,}\\
    f(t)&=f\left(x_0,\ldots,x_{n_\tau},p\right)\mbox{,}&&\\
    \tau_k(t)&=\tau_k\left(x_0,\ldots,x_{n_\tau},p\right)&&\mbox{\quad for $k=0,\ldots,n_\tau$,}\\
    \partial_{x,k}f(t)&=\frac{\partial}{\partial
      x_k}f\left(x_0,\ldots,x_{n_\tau},p\right)
    &&\mbox{\quad for $k=0,\ldots,n_\tau$,}\\
    \partial_{x,k}\tau_k(t)&=\frac{\partial}{\partial
      x_k}\tau_k\left(x_0,\ldots,x_{n_\tau},p\right)
    &&\mbox{\quad for $k=0,\ldots,n_\tau$,}\\
    f_k(t)&=f(x(t-\tau_{k,0}),\ldots,x(t-\tau_{k,n_\tau}),p)
    &&\mbox{\quad for $k=0,\ldots,n_\tau$,}\\
    \tau_{0,k}&=\tau_{k,0}=\tau_k(t)&&\mbox{\quad for $k=0,\ldots,n_\tau$,}\\
    \tau_{j,k}&=\tau_{j,0}+\tau_k(x(t-\tau_{j,0}),
    \ldots,x(t-\tau_{j,k-1}),p)&&\mbox{\quad for
      $k,j=1,\ldots,n_\tau$.}
  \end{aligned}
\end{equation}
The rows of the argument \mlvar{xx} for \mlvar{sys_rhs_SD_POfold}
consist of $x(\cdot)$ and $v(\cdot)$. The parameter array is extended
by $\beta$ and $T_\mt{copy}$ (a copy of the period $T$, the relation
$T=T_\mt{copy}$ is enforced in \mlvar{sys_cond_POfold}). The two
differential equations in \eqref{eq:sd:pofold} are augmented with the
phase condition built into DDE-BifTool, and additional system
conditions. The additional $3+n_\tau(n_\tau+1)/2$ system conditions
implemented in \blist{sys_cond_POfold} are identical to the extra
conditions of the constant-delay case
\eqref{eq:foldvtv1}--\eqref{eq:fold:tcopy}:
\begin{align}
  \label{eq:sdfoldvtv1}
  0&=\int_0^1 v(t)^Tv(t)\d t+\beta^2-1\\
  0&=\int_0^1 \dot x(t)^Tv(t)\d t \label{eq:sdfold:phas1}\\
  0&=\blist{point.parameter(npar+2)-point.period}\mbox{.} 
  \label{eq:sdfold:tcopy}
\end{align}
As the additional delays are not parameters, condition
\eqref{eq:fold:taurel} is not present (controlled by the parameter
\mlvar{relations} of \blist{sys_cond_POfold} being empty).

\section{Period doublings and torus bifurcations --- state-dependent
  delays}
\label{sec:sd:tr}
The function \mlvar{sys_rhs_SD_TorusBif} extends the user-defined
nonlinear problem with a coupled pair of variational problems
\eqref{eq:vp}---\eqref{eq:vp:qk} restricted to $q=0$ and
$\beta=0$. The approach is identical to the extended system for
constant delays outlined in Section~\ref{sec:ext:tr}. However, it
includes the derivatives of the time-delays. The extended system for
torus bifurcations computes a pair of two functions $u(t)$ and $v(t)$
($u(t)+\i v(t)$ is the critical complex Floquet mode). Consequently,
the extended system for torus bifurcations with state-dependent delays
requires a pair of sequences $U_k$ and $V_k$ (instead of $V_k$ and
$B_k$ for the fold of periodic orbits). The full extended system for
torus bifurcations and period doublings is:
\begin{equation}
  \label{eq:sd:torus}
  \begin{aligned}
    \dot x(t)&=f\left(x_0,\ldots,x_{n_\tau},p\right)\mbox{,}\\
        \dot u(t)&= \frac{\pi\omega}{T}v(t)+
        \sum_{k=0}^{n_\tau}\partial_{x,k}f(t)U_k\mbox{,}\\
        \dot v(t)&= -\frac{\pi\omega}{T}u(t)+
        \sum_{k=0}^{n_\tau}\partial_{x,k}f(t)V_k\mbox{,\qquad where}\\
        U_0&=u(t)\mbox{,}\quad V_0=v(t)
        \mbox{,\quad and for $k=1\ldots,n_\tau$}\\
    %     \begin{bmatrix}
    %       U_k\\V_k
    %     \end{bmatrix}&=
    %     \begin{bmatrix}
    %     \cos\left(\frac{\pi\omega\tau_k(t)}{T}\right)\mbox{,}&
    %     \sin\left(\frac{\pi\omega\tau_k(t)}{T}\right)\\
    %     -\sin\left(\frac{\pi\omega\tau_k(t)}{T}\right)\mbox{,}&
    %     \cos\left(\frac{\pi\omega\tau_k(t)}{T}\right)
    %     \end{bmatrix}
    %     \begin{bmatrix}
    %       u(t-\tau_k(t))\\
    %       v(t-\tau_k(t))
    %     \end{bmatrix}-
    %     \begin{bmatrix}
    %       f_k(t)\sum_{j=0}^{k-1}\partial_{x,j}
    % \tau_k(t)U_j\\
    %       f_k(t)\sum_{j=0}^{k-1}\partial_{x,j}
    % \tau_k(t)V_j
    %     \end{bmatrix}
    %     -f_k(t)\sum_{j=0}^{k-1}\partial_{x,j}
    % \tau_k(t)
    % \begin{bmatrix}
    %   U_j\\V_j      
    % \end{bmatrix}\mbox{.}
        U_k&=\phantom{-}
        \cos\left(\frac{\pi\omega\tau_k(t)}{T}\right)u(t-\tau_k(t))
        +\sin\left(\frac{\pi\omega\tau_k(t)}{T}\right)v(t-\tau_k(t))
        -f_k(t)\sum_{j=0}^{k-1}\partial_{x,j}
    \tau_k(t)U_j\mbox{,}\\
        V_k&=-\sin\left(\frac{\pi\omega\tau_k(t)}{T}\right)u(t-\tau_k(t))+
        \cos\left(\frac{\pi\omega\tau_k(t)}{T}\right)v(t-\tau_k(t))
        -f_k(t)\sum_{j=0}^{k-1}\partial_{x,j}
    \tau_k(t)V_j\mbox{.}
  \end{aligned}
\end{equation}
System\eqref{eq:sd:torus} uses the same set \eqref{eq:sd:abbrevs} of
notations as the system for the fold of periodic orbits.  The
additional conditions and the initialization are identical to the constant-delay case in Section~\ref{sec:ext:tr}.

\bibliographystyle{unsrt} \bibliography{../manual}
\appendix


\end{document}
